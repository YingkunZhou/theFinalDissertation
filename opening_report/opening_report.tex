% Search for all the places that say "PUT SOMETHING HERE".

\documentclass[11pt]{article}
\usepackage{amsmath,textcomp,amssymb,geometry,graphicx,enumerate}
\usepackage{tikz}
\usepackage{algorithm}  
\usepackage{algorithmicx}  
\usepackage{algpseudocode}
\usepackage{mathtools}
\usepackage[utf8]{inputenc}
\usepackage{hyperref}
\usepackage{mathtools}
\usepackage{ctex}
\usetikzlibrary{automata,positioning, arrows}
\renewcommand{\algorithmicrequire}{\textbf{input:}}  
\renewcommand{\algorithmicensure}{\textbf{output:}}

\def\Name{Yingkun Zhou}  % Your name
\def\SID{2015K8009929023}  % Your student ID number

\def\le{\leqslant}
\def\logN{\log{}n}
\newcommand{\ro}[1]{\romannumeral #1}



\title{UCAS undergraduate opening report}
\author{\Name, \SID}
\date{}

\newenvironment{qparts}{\begin{enumerate}[{(}a{)}]}{\end{enumerate}}
\def\endproofmark{$\Box$}
\newenvironment{proof}{\par{\bf Proof}:}{\endproofmark\smallskip}

\textheight=9in
\textwidth=6.5in
\topmargin=-.75in
\oddsidemargin=0.25in
\evensidemargin=0.25in


\begin{document}
\maketitle
\section*{0.Opening}
我一直有个比喻在计算机领域,体系结构者扮演的是上帝的角色。上帝造万物,制规律,但却察觉不到。体系结构不就是如此吗,人们只关注其上跑的应用如否满足自己的需求,是否流畅(哪怕这个体系结构的ISA再烂,比如Intel),如果不是计算机系的人,根本不会关注其背后的规律,何种指令以何种方式被执行。而操作体统犹如国王,国王管理着一切,当然国王也被叫做天子,因为是和上帝紧紧绑定的。而当今世界的三大国家为Windows, MacOS, Linux. 一般人们也不会感受到国家的存在,但是如果跨国的话,有时候就能够鲜明的感受到不同国家的区别。软件犹如国家的臣民,国家的机构必须遵从国家的规章制度,遵从内核调度,采用系统调用,跨国公司也概莫能外。编译器则是使者,沟通着天子,臣民与上帝。这个也很好理解,高级语言要转化为机器码才能够被执行。本科的阶段没有发论文的打算,一是体系结构这个领域十年磨一剑,二是想在本科阶段做做基础性的工作,打好以后研究的基础。

\section*{1.Background}
体系结构领域出现了一股不可忽视的力量,其影响力将不啻OS领域的Linux,这就是risc-v。而我选择risc-v作为本科毕业设计的大的框架背景也是处于如下几个方面的考虑。
\begin{itemize}
	\item 在今年的龙芯杯中,自己设计的CPU以及在上面做的software stack的工作虽然侥幸获得了第一,但是
\end{itemize}

\section*{2.language and logic}
首先语言的更高层次化是为了代码的简洁与复用,换言之就是由编译器来做琐碎的事情,从而解放人去更多的思考逻辑的事。但是语言的更高层次化也不应该提高代码的逻辑复杂密度。不得不承认,从C到C++以及Java,代码的逻辑复杂密度有所提高,C是简单的二层逻辑--global \& function local; 面向对象则是各种类的继承,抽象,方法的多态辅以各种一时难以理解的$ \lambda $-函数,iteration迭代,代码结构难以trace。如果代码框架是自己写的,那自然非常happy,因为编译器帮你做了很多琐碎的事情,但是旁人如果想修改一开始就非常头疼(举个最简单的例子,C代码中如果你想标准输出,想都不用想直接调用printf,但是Java中呢,你就必须得知道print函数是在哪个class中,那个class又是在哪个package中,像我初学Java的时候真的是每一个想调用print的时候就必定会查google;还有一个毛病就是C可以GDB,当然C++也可以GDB,但调起来就有些头疼了,有时候你根本不知道代码会什么会跳到那里(就是前面说的难以trace),Java的话貌似都没有GDB这么一说,每次我都用print大法)。

言归正传,但是面向对象语言所付出的这些代价都是值得的,尤其是在硬件逻辑的编写领域。为什么这么说呢?现在任给你一张CPU的架构图,一定是一些方框的逻辑组合,这些方框又是什么呢?就是模块,既然是模块,那么目前最好的刻画方式就是面向对象的语言。我一直很好奇为什么我们能够忍受这么多年的Verilog,难道高层的语言解决不了的电路的逻辑吗?一开始我是这么认为的,因为sequential circuit 的刻画就像是盘亘在高级语言上的一座大山, 如果仔细一想,哪有语言想要去刻画reg。但是直到接触了chisel,方才惊呼伯克利体系结构的这个上帝高实在是高。让人钦佩的是这个上帝同时精通使者的工作,使得在刻画逻辑和规律的时候更加游刃有余。

有了面向对象的技术,我们可以充分利用类的继承来简化繁琐的模块之间的接口与连线,更好的进行代码复用,不可综合的逻辑电路再也不用困扰我们,没有组合环,没有 asynchronous sequential logic circuit. 伯克利的设计者甚至把Java里面的test哲学也搬进了电路设计中(基于每个类都予以测试),同时还简化了simulation的逻辑的编写,不需要自己设计时钟(隔多少秒时钟打过一拍),用一个简简单单的step(1)就是打过一拍,事实上这也这是我们思考电路时的逻辑。
\def\St{\State}
\DeclarePairedDelimiter\ceil{\lceil}{\rceil}
\DeclarePairedDelimiter\floor{\lfloor}{\rfloor}


\end{document}
