% Search for all the places that say "PUT SOMETHING HERE".

\documentclass[11pt]{article}
\usepackage{amsmath,textcomp,amssymb,geometry,graphicx,enumerate}
\usepackage{tikz}
\usepackage{algorithm}  
\usepackage{algorithmicx}  
\usepackage{algpseudocode}
\usepackage{mathtools}
\usepackage[utf8]{inputenc}
\usepackage{hyperref}
\usepackage{mathtools}
\usepackage{ctex}
\usetikzlibrary{automata,positioning, arrows}
\renewcommand{\algorithmicrequire}{\textbf{input:}}  
\renewcommand{\algorithmicensure}{\textbf{output:}}

\def\Name{Yingkun Zhou}  % Your name
\def\SID{2015K8009929023}  % Your student ID number

\def\le{\leqslant}
\def\logN{\log{}n}
\newcommand{\ro}[1]{\romannumeral #1}



\title{多发射乱序CPU的RTL级实现与验证}
\author{\Name, \SID}
\date{}

\newenvironment{qparts}{\begin{enumerate}[{(}a{)}]}{\end{enumerate}}
\def\endproofmark{$\Box$}
\newenvironment{proof}{\par{\bf Proof}:}{\endproofmark\smallskip}

\textheight=9in
\textwidth=6.5in
\topmargin=-.75in
\oddsidemargin=0.25in
\evensidemargin=0.25in


\begin{document}
\maketitle
\textbf{key word: }RISC-V, MIPS, CPU, out-of-order,multi-issue

乱序多发射的CPU设计并不容易,而且比设计更加困难的是调试,如何保证复杂执行的CPU能够有条不紊,毫无错误的运行比如像Linux操作系统这样的大型软件更加难上加难。参考RISC-V开源部分的资料,采用的方案如下:
\begin{enumerate}
	\item 对于单一时钟,同步复位的电路,采用chisel来编写,考虑到其对于逻辑的刻画效率比verilog更高,而且对于可综合电路的描述能够做到和verilog同样的精确
	\item 其次乱序的CPU如今也有开源的BOOM代码实现,可以很好的参考借鉴
	\item RISC-V目前有一整套的调试手段可以学习借用
	\item RISC-V有整套的软件栈可以运行在自行设计的CPU之上
\end{enumerate}

综上,计划的第一步是实现一个基于RISC-V ISA的乱序双发射的CPU,调试完毕能够跑通简单的测试程序。本人也深知这绝非易事(希望能够比较顺利完成)。接着在时间允许的情况下做操作系统的移植,继而能够支持多核。注意到一开始的设计必须考虑到微结构和ISA之间的解耦合,这样将RISC-V改成MIPS工作量会降低很多。得到MIPS版本的CPU目的在于采用相同的微结构可以客观的对比两个ISA的优劣。同时作为工作的一部分,性能的优化以及设计空间的搜索必不可少,期望能够在跑相同benchmark的情况下持平甚至超过ARM的Cortex A53的水平。

\end{document}
