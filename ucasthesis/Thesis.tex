%---------------------------------------------------------------------------%
%-                                                                         -%
%-                           LaTeX Template                                -%
%-                                                                         -%
%---------------------------------------------------------------------------%
%- Copyright (C) Huangrui Mo <huangrui.mo@gmail.com> 
%- This is free software: you can redistribute it and/or modify it
%- under the terms of the GNU General Public License as published by
%- the Free Software Foundation, either version 3 of the License, or
%- (at your option) any later version.
%---------------------------------------------------------------------------%
%->> Document class declaration
%---------------------------------------------------------------------------%
\documentclass[doublesided]{Style/ucasthesis}%
%- Multiple optional arguments:
%- [<singlesided|doublesided|printcopy>]% set one or two sided eprint or print
%- [draftversion]% show draft version information
%- [fontset=<fandol|...>]% specify font set to replace automatic detection
%- [scheme=plain]% thesis writing of international students
%- [standard options for ctex book class: draft|paper size|font size|...]%
%---------------------------------------------------------------------------%
%->> Document settings
%---------------------------------------------------------------------------%
\usepackage[authoryear,myhdr,list]{Style/artratex}% document settings
%- usage: \usepackage[option1,option2,...,optionN]{artratex}
%- Multiple optional arguments:
%- [bibtex|biber]% set bibliography processor and package
%- [<numbers|super|authoryear|alpha>]% set citation and reference style
%- <numbers>: textual: Jones [1]; parenthetical: [1]
%- <super>: textual: Jones superscript [1]; parenthetical: superscript [1]
%- <authoryear>: textual: Jones (1995); parenthetical: (Jones, 1995)
%- <alpha>: textual: not available; parenthetical: [Jon95]
%- [geometry]% reconfigure page layout via geometry package
%- [lscape]% provide landscape layout environment
%- [myhdr]% enable header and footer via fancyhdr package
%- [color]% provide color support via xcolor package
%- [background]% enable page background
%- [tikz]% provide complex diagrams via tikz package
%- [table]% provide complex tables via ctable package
%- [list]% provide enhanced list environments for algorithm and coding
%- [math]% enable some extra math packages
\usepackage{Style/artracom}% user defined commands
\usepackage{color}
\usepackage{adjustbox}
\usepackage{tikz}
\usetikzlibrary{automata,positioning, arrows}
\usetikzlibrary{shapes}
\usetikzlibrary{positioning, quotes}
\usepackage[usestackEOL]{stackengine}
\lstdefinelanguage{scala}{
	morekeywords={abstract,case,catch,class,def,%
		do,else,extends,false,final,finally,%
		for,if,implicit,import,match,mixin,%
		new,null,object,override,package,%
		private,protected,requires,return,sealed,%
		super,this,throw,trait,true,try,%
		type,val,var,while,with,yield},
	sensitive=true,
	morecomment=[l]{//},
	morecomment=[n]{/*}{*/},
	morestring=[b]",
	morestring=[b]',
	morestring=[b]"""
}

\usepackage{pgfplotstable}
\pgfplotsset{
	percentage plot/.style={
		point meta=explicit,
		nodes near coords align=vertical,
		yticklabel=\pgfmathprintnumber{\tick}\,$\%$,
		ymin=0,
		ymax=100,
		enlarge y limits={upper,value=0},
		visualization depends on={y \as \originalvalue}
	},
	percentage series/.style={
		table/y expr=\thisrow{#1},table/meta=#1
	}
}

\definecolor{dkgreen}{rgb}{0,0.6,0}
\definecolor{gray}{rgb}{0.5,0.5,0.5}
\definecolor{mauve}{rgb}{0.58,0,0.82}

% Default settings for code listings
\lstset{frame=tb,
	language=scala,
	aboveskip=0mm,
	belowskip=-5mm,
	showstringspaces=false,
	columns=fixed, % basewidth=\mybasewidth,
	basicstyle={\footnotesize\ttfamily},
	numbers=none,
	numberstyle=\footnotesize\color{gray},
	% identifierstyle=\color{red},
	keywordstyle=\color{blue},
	commentstyle=\color{dkgreen},
	stringstyle=\color{mauve},
	frame=single,
	breaklines=true,
	breakatwhitespace=true,
	procnamekeys={def, val, var, class, trait, object, extends},
	procnamestyle=\ttfamily\color{red},
	tabsize=2
}

\lstnewenvironment{scala}[1][]
{\lstset{language=scala,frame=none,#1}}
{}

\usepackage{pgfplots}
\definecolor{bblue}{HTML}{4F81BD}
\definecolor{rred}{HTML}{C0504D}
\definecolor{ggreen}{HTML}{9BBB59}
\definecolor{ppurple}{HTML}{9F4C7C}
\usetikzlibrary{patterns}
\definecolor{rosso}{RGB}{220,57,18}
\definecolor{giallo}{RGB}{255,153,0}
\definecolor{blu}{RGB}{102,140,217}
\definecolor{verde}{RGB}{16,150,24}
\definecolor{viola}{RGB}{153,0,153}
\definecolor{pink}{RGB}{220,100,100}

\makeatletter

\tikzstyle{chart}=[
legend label/.style={font={\scriptsize},anchor=west,align=left},
legend box/.style={rectangle, draw, minimum size=5pt},
axis/.style={black,semithick,->},
axis label/.style={anchor=east,font={\tiny}},
]

\tikzstyle{bar chart}=[
chart,
bar width/.code={
	\pgfmathparse{##1/2}
	\global\let\bar@w\pgfmathresult
},
bar/.style={very thick, draw=white},
bar label/.style={font={\bf\small},anchor=north},
bar value/.style={font={\footnotesize}},
bar width=.75,
]

\tikzstyle{pie chart}=[
chart,
slice/.style={line cap=round, line join=round, very thick,draw=white},
pie title/.style={font={\bf}},
slice type/.style 2 args={
	##1/.style={fill=##2},
	values of ##1/.style={}
}
]

\pgfdeclarelayer{background}
\pgfdeclarelayer{foreground}
\pgfsetlayers{background,main,foreground}

\newcommand{\pie}[3][]{
	\begin{scope}[#1]
		\pgfmathsetmacro{\curA}{90}
		\pgfmathsetmacro{\r}{1}
		\def\c{(0,0)}
		\node[pie title] at (90:1.3) {#2};
		\foreach \v/\s in{#3}{
			\pgfmathsetmacro{\deltaA}{\v/100*360}
			\pgfmathsetmacro{\nextA}{\curA + \deltaA}
			\pgfmathsetmacro{\midA}{(\curA+\nextA)/2}
			
			\path[slice,\s] \c
			-- +(\curA:\r)
			arc (\curA:\nextA:\r)
			-- cycle;
			%\pgfmathsetmacro{\d}{max((\deltaA * -(.5/50) + 1) , .5)}
			
			% Tests on de value of \deltaA allowing to assign different values to the style parameters
			\pgfmathparse{\deltaA > 50 ? \r/2 : \r}
			\edef\d{\pgfmathresult}
			% If the pie portion is 5% or lower the label is plotted outside of the pie else it is plotted at half the radius
			\pgfmathparse{\deltaA > 50 ? \d : \r+0.5/\v}
			\edef\dd{\pgfmathresult}
			% If the pie portion is 5% or lower the path is plotted
			\pgfmathparse{\deltaA > 50 ? "none" : "black"}
			\edef\linkcolor{\pgfmathresult}
			\pgfmathparse{\deltaA > 50 ? "none" : "white"}
			\edef\fillcolor{\pgfmathresult}
			% These last parameters allow to place the label differently around the node depending on the value of \deltaA
			\pgfmathparse{\deltaA > 50 ? 0 : 1}
			\edef\pos{\pgfmathresult}
			\pgfmathparse{\deltaA > 50 ? 0 : 5}
			\edef\shiftA{\pgfmathresult}
			\pgfmathparse{\deltaA > 50 ? "center" : "west"}
			\edef\anchor{\pgfmathresult}
			
			\begin{pgfonlayer}{foreground}
				%\path \c -- node[pos=\d,pie values,values of \s]{$\v\%$} +(\midA:\r);
				\path[draw=\linkcolor] (\midA:\d) --  (\midA:\dd) -- node[pos=\pos,pie values,values of \s,draw=none,fill=\fillcolor,inner sep=0pt,anchor=  \anchor]{$\v\%$}(\midA-\shiftA:\dd);
			\end{pgfonlayer}
			
			\global\let\curA\nextA
		}
	\end{scope}
}

\newcommand{\legend}[2][]{
	\begin{scope}[#1]
		\path
		\foreach \n/\s in {#2}
		{
			++(0,-10pt) node[\s,legend box] {} +(5pt,0) node[legend label] {\n}
		}
		;
	\end{scope}
}
%---------------------------------------------------------------------------%
%->> Document inclusion
%---------------------------------------------------------------------------%
%\includeonly{Tex/Chap_1,...,Tex/Chap_N}% selected files compilation
%---------------------------------------------------------------------------%
%->> Document content
%---------------------------------------------------------------------------%
\begin{document}
	%-
	%-> Frontmatter: title page, abstract, content list, symbol list, preface
	%-
	\frontmatter% initialize the environment
	%---------------------------------------------------------------------------%
%->> 封面信息及生成
%---------------------------------------------------------------------------%
%-
%-> 中文封面信息
%-
\confidential{}% 密级:只有涉密论文才填写
\schoollogo{scale=0.095}{ucas_logo}% 校徽
\title{乱序双发射处理器的自主设计实现}% 论文中文题目
\author{周盈坤}% 论文作者
\advisor{胡伟武~研究员~中国科学院计算技术研究所}% 指导教师:姓名 专业技术职务 工作单位
\advisorsec{}% 指导老师附加信息 或 第二指导老师信息
\degree{学士}% 学位:学士、硕士、博士
\degreetype{工学}% 学位类别:理学、工学、工程、医学等
\major{计算机体系结构}% 二级学科专业名称
\institute{计算机科学与技术学院}% 院系名称
\chinesedate{2019~年~6~月}% 毕业日期:夏季为6月、冬季为12月
%-
%-> 英文封面信息
%-
\englishtitle{Independent Design of the Two-wide Superscalar \\ Microprocessor with Out-of-order Execution}% 论文英文题目
\englishauthor{Zhou Yingkun}% 论文作者
\englishadvisor{Supervisor: Professor Hu Weiwu}% 指导教师
\englishdegree{Bachelor}% 学位:Bachelor, Master, Doctor。封面格式将根据英文学位名称自动切换,请确保拼写准确无误
\englishdegreetype{Computer Science}% 学位类别:Philosophy, Natural Science, Engineering, Economics, Agriculture 等
\englishthesistype{thesis}% 论文类型: thesis, dissertation
\englishmajor{Computer Architecture}% 二级学科专业名称
\englishinstitute{Institute of Computing Technology, Chinese Academy of Sciences}% 院系名称
\englishdate{June, 2019}% 毕业日期:夏季为June、冬季为December
%-
%-> 生成封面
%-
\maketitle% 生成中文封面
\makeenglishtitle% 生成英文封面
%-
%-> 作者声明
%-
\makedeclaration% 生成声明页
%-
%-> 中文摘要
%-
\chapter*{摘\quad 要}\chaptermark{摘\quad 要}% 摘要标题
\setcounter{page}{1}% 开始页码
\pagenumbering{Roman}% 页码符号
本文旨在分析探讨采用乱序、多发射、超标量技术的高性能通用处理器结构设计,以及对其性能的考量、权衡与取舍,并自主设计出一款能够运行嵌入式程序的乱序双发射处理器。为了有效的控制设计的复杂度,在较短的时间内设计出一款能够运行程序的乱序双发射处理器就必须有所简化。最后简化的方案如下:首先ISA指令集在32位架构的前提下选择了目前最为精简的RISC-V中已经被冻结的RV32I指令集;其次编写的语言采用由加州大学伯克利分校开发的Chisel语言,来提高设计的效率以及有效控制设计的复杂度。最后对比同样是自己设计的单发射五级静态流水和双发射五级静态流水架构,双发射乱序流水架构运行程序所需的周期数明显减少,从而性能有了显著的提高。同时,对benchmark中的各个性能测试程序自身指令特点和在不同架构下的执行行为作了细致的比较,量化地分析了影响性能的因素以及这些因素是如何抑制或者提高处理器性能的。

\keywords{微处理器,乱序,双发射,RISC-V}% 中文关键词
%-
%-> 英文摘要
%-
\chapter*{Abstract}\chaptermark{Abstract}% 摘要标题

This paper aims to build a totally self-designed microprocessor with out-of-order, superscalar and multi-issue micro-architecture features to enhance performances of executing programs especially in embedded system area. However, the microprocessor is very hard to implement. In order to control the complexity as will as being able to accomplish the design in this half year, some simplifications have been taken and the final design is as follows. Firstly, as one of the simplest 32-bit ISA, RV32I which is frozen by RISC-V has been chosen to be supported by the microprocessor. Secondly, because of the great ability of complexity control, Chisel which is designed and maintained by University of California, Berkeley has been chosen to be the programming language to design the microprocessor efficiently. In the end, the two-wide superscalar microprocessor with out-of-order execution has been developed successfully. And then, compared with single-issue and dual-issue five stage static pipeline processors, the execution performance of the out-of-order processor improved apparently. Finally, by analysis the dynamic trace of every programs after the processors executing them, several elements are found impacting the performance.

\englishkeywords{Microprocessor, Out-of-order, Two-wide Superscalar, RISC-V}% 英文关键词
%---------------------------------------------------------------------------%
% title page, abstract, dedication
	{% content list region
		\linespread{1.2}% local line space
		%\intotoc{\contentsname}% add link to contents table and bookmark
		\tableofcontents% contents catalog
		%\intotoc{\listfigurename}% add link to contents table and bookmark
		\listoffigures% figures catalog
		%\intotoc{\listtablename}% add link to contents table and bookmark
		\listoftables% tables catalog
	}
	\chapter*{符号列表}
\chaptermark{符号列表}

% \section*{字符}
% \nomenclatureitem[\textbf{Unit}]{\textbf{Symbol}}{\textbf{Description}}
% \nomenclatureitem[$\Unit{m^{2} \cdot s^{-2} \cdot K^{-1}}$]{$R$}{the gas constant}
% \nomenclatureitem[$\Unit{m^{2} \cdot s^{-2} \cdot K^{-1}}$]{$C_v$}{specific heat capacity at constant volume}
% \nomenclatureitem[$\Unit{m^{2} \cdot s^{-2} \cdot K^{-1}}$]{$C_p$}{specific heat capacity at constant pressure}
% \nomenclatureitem[$\Unit{m^{2} \cdot s^{-2}}$]{$E$}{specific total energy}
% \nomenclatureitem[$\Unit{m^{2} \cdot s^{-2}}$]{$e$}{specific internal energy}
% \nomenclatureitem[$\Unit{m^{2} \cdot s^{-2}}$]{$h_T$}{specific total enthalpy}
% \nomenclatureitem[$\Unit{m^{2} \cdot s^{-2}}$]{$h$}{specific enthalpy}
% \nomenclatureitem[$\Unit{kg \cdot m \cdot s^{-3} \cdot K^{-1}}$]{$k$}{thermal conductivity}
% \nomenclatureitem[$\Unit{kg \cdot m^{-1} \cdot s^{-2}}$]{$S_{ij}$}{deviatoric stress tensor}
% \nomenclatureitem[$\Unit{kg \cdot m^{-1} \cdot s^{-2}}$]{$\tau_{ij}$}{viscous stress tensor}
% \nomenclatureitem[$\Unit{1}$]{$\delta_{ij}$}{Kronecker tensor}
% \nomenclatureitem[$\Unit{1}$]{$I_{ij}$}{identity tensor}

% \section*{算子}
% \nomenclatureitem{\textbf{Symbol}}{\textbf{Description}}
% \nomenclatureitem{$\Delta$}{difference}
% \nomenclatureitem{$\nabla$}{gradient operator}
% \nomenclatureitem{$\delta^{\pm}$}{upwind-biased interpolation scheme}

\section*{缩写}
\nomenclatureitem{IPC}{Instruction Per Cycle}
\nomenclatureitem{CAM}{Content Addressable Memory}
\nomenclatureitem{RAM}{Ramdom Access Memory}
\nomenclatureitem{LW}{Load and Store instruction/request}
\nomenclatureitem{ISA}{Instruction set architecturey}
\nomenclatureitem{TLB}{Translation lookaside buffer}
\nomenclatureitem{FSM}{Finite state machine}
\nomenclatureitem{BOOM}{The Berkeley Out-of-Order RISC-V Processor}
\nomenclatureitem{PC}{The program counter}
\nomenclatureitem{LDQ}{The load queue}
\nomenclatureitem{STQ}{The store queue}
\nomenclatureitem{ALU}{Arithmetic logic unit}
\nomenclatureitem{MIPS}{Microprocessor without Interlocked Pipelined Stages}
\nomenclatureitem{RISC}{Reduced instruction set computer}
\nomenclatureitem{ROB}{Reorder Buffer}
\nomenclatureitem{BTB}{Branch Target Buffer}
\nomenclatureitem{RAS}{Return Address Stack}
\nomenclatureitem{FPGA}{Field-Programmable Gate Array}% list of symbols, preface content
	%-
	%-> Mainmatter
	%-
	\mainmatter% initialize the environment
	\input{Tex/Mainmatter}% main content
	%-
	%-> Appendix
	%-
	\cleardoublepage%
	\appendix% initialize the environment
	%\input{Tex/Appendix}% appendix content
	%-
	%-> Backmatter: bibliography, glossary, index
	%-
	\backmatter% initialize the environment
	\intotoc{\bibname}% add link to contents table and bookmark
	\bibliography{Biblio/ref}% bibliography
	\input{Tex/Backmatter}% other information
\end{document}
%---------------------------------------------------------------------------%